%%%%%%%%%%%%%%%%%%%%%%%%%%%%%%%%%%%%%%%%%%%%%%%%%%%%%%%%%%%%%%%%%%%%%%%%
%%%%%%%%%%%%%%%%%%%%%% Simple LaTeX CV Template %%%%%%%%%%%%%%%%%%%%%%%%
%%%%%%%%%%%%%%%%%%%%%%%%%%%%%%%%%%%%%%%%%%%%%%%%%%%%%%%%%%%%%%%%%%%%%%%%

%%%%%%%%%%%%%%%%%%%%%%%%%%%%%%%%%%%%%%%%%%%%%%%%%%%%%%%%%%%%%%%%%%%%%%%%
%% NOTE: If you find that it says                                     %%
%%                                                                    %%
%%                           1 of ??                                  %%
%%                                                                    %%
%% at the bottom of your first page, this means that the AUX file     %%
%% was not available when you ran LaTeX on this source. Simply RERUN  %%
%% LaTeX to get the ``??'' replaced with the number of the last page  %%
%% of the document. The AUX file will be generated on the first run   %%
%% of LaTeX and used on the second run to fill in all of the          %%
%% references.                                                        %%
%%%%%%%%%%%%%%%%%%%%%%%%%%%%%%%%%%%%%%%%%%%%%%%%%%%%%%%%%%%%%%%%%%%%%%%%

%%%%%%%%%%%%%%%%%%%%%%%%%%%% Document Setup %%%%%%%%%%%%%%%%%%%%%%%%%%%%

% Don't like 10pt? Try 11pt or 12pt
\documentclass[11pt]{article}
\usepackage{engord}
% This is a helpful package that puts math inside length specifications
\usepackage{calc}

% Simpler bibsection for CV sections
% (thanks to natbib for inspiration)
\makeatletter
\newlength{\bibhang}
\setlength{\bibhang}{1em}
\newlength{\bibsep}
 {\@listi \global\bibsep\itemsep \global\advance\bibsep by\parsep}
\newenvironment{bibsection}%
        {\vspace{-\baselineskip}\begin{list}{}{%
       \setlength{\leftmargin}{\bibhang}%
       \setlength{\itemindent}{-\leftmargin}%
       \setlength{\itemsep}{\bibsep}%
       \setlength{\parsep}{\z@}%
        \setlength{\partopsep}{0pt}%
        \setlength{\topsep}{0pt}}}
        {\end{list}\vspace{-.6\baselineskip}}
\makeatother

% Layout: Puts the section titles on left side of page
\reversemarginpar

%
%         PAPER SIZE, PAGE NUMBER, AND DOCUMENT LAYOUT NOTES:
%
% The next \usepackage line changes the layout for CV style section
% headings as marginal notes. It also sets up the paper size as either
% letter or A4. By default, letter was used. If A4 paper is desired,
% comment out the letterpaper lines and uncomment the a4paper lines.
%
% As you can see, the margin widths and section title widths can be
% easily adjusted.
%
% ALSO: Notice that the includefoot option can be commented OUT in order
% to put the PAGE NUMBER *IN* the bottom margin. This will make the
% effective text area larger.
%
% IF YOU WISH TO REMOVE THE ``of LASTPAGE'' next to each page number,
% see the note about the +LP and -LP lines below. Comment out the +LP
% and uncomment the -LP.
%
% IF YOU WISH TO REMOVE PAGE NUMBERS, be sure that the includefoot line
% is uncommented and ALSO uncomment the \pagestyle{empty} a few lines
% below.
%

%% Use these lines for letter-sized paper
\usepackage[paper=a4paper,
            %includefoot, % Uncomment to put page number above margin
            marginparwidth=1.1in,     % Length of section titles
            marginparsep=.05in,       % Space between titles and text
            top=1.1in, bottom=0.3in, left=0.60in, right=0.60in,               % 1 inch margins
            includemp]{geometry}

%% Use these lines for A4-sized paper
%\usepackage[paper=a4paper,
%            %includefoot, % Uncomment to put page number above margin
%            marginparwidth=30.5mm,    % Length of section titles
%            marginparsep=1.5mm,       % Space between titles and text
%            margin=25mm,              % 25mm margins
%            includemp]{geometry}

%% More layout: Get rid of indenting throughout entire document
\setlength{\parindent}{0in}

%% This gives us fun enumeration environments. compactitem will be nice.
\usepackage{paralist}

%% Reference the last page in the page number
%
% NOTE: comment the +LP line and uncomment the -LP line to have page
%       numbers without the ``of ##'' last page reference)
%
% NOTE: uncomment the \pagestyle{empty} line to get rid of all page
%       numbers (make sure includefoot is commented out above)
%
\usepackage{fancyhdr,lastpage}
\pagestyle{fancy}
\pagestyle{empty}      % Uncomment this to get rid of page numbers
\fancyhf{}\renewcommand{\headrulewidth}{0pt}
\fancyfootoffset{\marginparsep+\marginparwidth}
\newlength{\footpageshift}
\setlength{\footpageshift}
          {0.5\textwidth+0.5\marginparsep+0.5\marginparwidth-2in}
\lfoot{\hspace{\footpageshift}%
       \parbox{4in}{\, \hfill %
                    \arabic{page} of \protect\pageref*{LastPage} % +LP
%                    \arabic{page}                               % -LP
                    \hfill \,}}

% Finally, give us PDF bookmarks
\usepackage{color,hyperref}
\definecolor{darkblue}{rgb}{0.0,0.0,0.3}
\hypersetup{colorlinks,breaklinks,
            linkcolor=darkblue,urlcolor=darkblue,
            anchorcolor=darkblue,citecolor=darkblue}

%%%%%%%%%%%%%%%%%%%%%%%% End Document Setup %%%%%%%%%%%%%%%%%%%%%%%%%%%%


%%%%%%%%%%%%%%%%%%%%%%%%%%% Helper Commands %%%%%%%%%%%%%%%%%%%%%%%%%%%%

% The title (name) with a horizontal rule under it
%
% Usage: \makeheading{name}
%
% Place at top of document. It should be the first thing.
\newcommand{\makeheading}[1]%
        {\hspace*{-\marginparsep minus \marginparwidth}%
         \begin{minipage}[t]{\textwidth+\marginparwidth+\marginparsep}%
                {\large \bfseries #1}\\[-0.15\baselineskip]%
                 \rule{\columnwidth}{1pt}%
         \end{minipage}}

% The section headings
%
% Usage: \section{section name}
%
% Follow this section IMMEDIATELY with the first line of the section
% text. Do not put whitespace in between. That is, do this:
%
%       \section{My Information}
%       Here is my information.
%
% and NOT this:
%
%       \section{My Information}
%
%       Here is my information.
%
% Otherwise the top of the section header will not line up with the top
% of the section. Of course, using a single comment character (%) on
% empty lines allows for the function of the first example with the
% readability of the second example.
\renewcommand{\section}[2]%
        {\pagebreak[2]\vspace{0.80\baselineskip}%
         \phantomsection\addcontentsline{toc}{section}{#1}%
         \hspace{0in}%
         \marginpar{
         \raggedright \scshape #1}#2}

% An itemize-style list with lots of space between items
\newenvironment{outerlist}[1][\enskip\textbullet]%
        {\begin{itemize}[#1]}{\end{itemize}%
         \vspace{-.6\baselineskip}}

% An environment IDENTICAL to outerlist that has better pre-list spacing
% when used as the first thing in a \section
\newenvironment{lonelist}[1][\enskip\textbullet]%
        {\vspace{-\baselineskip}\begin{list}{#1}{%
        \setlength{\partopsep}{0pt}%
        \setlength{\topsep}{0pt}}}
        {\end{list}\vspace{-.6\baselineskip}}

% An itemize-style list with little space between items
\newenvironment{innerlist}[1][\enskip\textbullet]%
        {\begin{compactitem}[#1]}{\end{compactitem}}

% An environment IDENTICAL to innerlist that has better pre-list spacing
% when used as the first thing in a \section
\newenvironment{loneinnerlist}[1][\enskip\textbullet]%
        {\vspace{-\baselineskip}\begin{compactitem}[#1]}
        {\end{compactitem}\vspace{-.6\baselineskip}}

% To add some paragraph space between lines.
% This also tells LaTeX to preferably break a page on one of these gaps
% if there is a needed pagebreak nearby.
\newcommand{\blankline}{\quad\pagebreak[2]}

% Uses hyperref to link DOI
\newcommand\doilink[1]{\href{http://dx.doi.org/#1}{#1}}
\newcommand\doi[1]{doi:\doilink{#1}}


%%%%%%%%%%%%%%%%%%%%%%%% End Helper Commands %%%%%%%%%%%%%%%%%%%%%%%%%%%

%%%%%%%%%%%%%%%%%%%%%%%%% Begin CV Document %%%%%%%%%%%%%%%%%%%%%%%%%%%%

\begin{document}
\makeheading{\LARGE Mr. Jialu Liu}

\section{Contact Information}
%
% NOTE: Mind where the & separators and \\ breaks are in the following
%       table.
%
% ALSO: \rcollength is the width of the right column of the table
%       (adjust it to your liking; default is 1.85in).
%
\newlength{\rcollength}\setlength{\rcollength}{2.28in}%
%
\begin{tabular}[t]{@{}p{\textwidth-\rcollength}p{\rcollength}}
%
Rm. 276, No. 32 Dormitory, Yuquan Campus       & \textit{Cell:} (86) 13732224597 \\
{Zhejiang University}           & \textit{E-mail:}
\href{mailto:remenberl@gmail.com}{remenberl@gmail.com}\\
Hangzhou, 310027, P.R. China    & \textit{WWW:}
\href{http://www.relau.com/jialuliu}{www.relau.com/jialuliu}\\
\end{tabular}

\section{Research Interests}
%
Machine Learning, especially interested in developing and applying machine learning algorithms in the field of Information Retrieval, Data Mining and Computer Vision.


\section{Education}
Candidate for Bachelor in Engineering (Computer Science), Expected June 2011\\
        \href{http://www.cs.zju.edu.cn/}
             {College of Computer Science and Technology},
        \href{http://www.zju.edu.cn/english/}{Zhejiang University}
        \begin{innerlist}
        \item Overall GPA: 3.92/4.0 (88.7/100)
        \item Major GPA: 3.96/4.0 (90.4/100)
        \end{innerlist}



\section{Publications}
    \textbf{Jialu Liu}, Deng Cai and Xiaofei He, “Gaussian Mixture Model with
    Local Consistency“, \emph{The Twenty-Fourth AAAI Conference on
    Artificial Intelligence} (AAAI'10, Oral \& Poster)

    \vspace{1.7mm}
    Wei Xue, \textbf{Jialu Liu}, Deng Cai, “Distance Metric Learning for Image Annotation“, \emph{IEEE Transaction on Image Processing} (to be submitted)


\section{Research Experience}
\textbf{Research Assis. of Associate Prof. Deng Cai, State Key Lab. of CAD \& CG
    }

\vspace{0.25\baselineskip}
\textit{Distance Metric Learning for Image Annotation}\hspace{3.522cm}2010/08 - Present
\vspace{0.08\baselineskip}
    \begin{innerlist}
            \item Handling annotation problem using image retrieval and metric learning techniques.
    \end{innerlist}

\vspace{0.4\baselineskip}
\textit{Saliency Based Trimap Segmentation}\hspace{5.364cm}2010/08 - Present
\vspace{0.1\baselineskip}
    \begin{innerlist}
            \item Proposing a new algorithm for trimap segmentation: pixels are firstly classified using Grabcut, then an energy minimization problem is solved with saliency initialization.

    \end{innerlist}

\vspace{0.4\baselineskip}
\textit{Local Consistent Gaussian Mixture Model}%
        \hfill 2009/08 - 2010/01
\vspace{0.08\baselineskip}
    \begin{innerlist}

            \item Focused on considering the intrinsic geometry of conditional probability
                distribution along the geodesics of data manifold
                by incorporating a smoothing regularizer into the log-likelihood objective
                function of the traditional Gaussian Mixture Model.

    \end{innerlist}

\vspace{0.5\baselineskip}
\textbf{Research Assis. of Prof. Zhigeng Pan, State Key Lab. of CAD \& CG}

\vspace{0.25\baselineskip}
 \textit{Hand Gestures Recognition}%
        \hfill 2009/07 - 2009/11
\vspace{0.08\baselineskip}
    \begin{innerlist}
            \item Succeeded in recognizing static and dynamic hand gestures under complicated background, adopting trained Gaussian Models of human skin-color.

    \end{innerlist}


\section{Engineering Experience}
  \textit{Face Recognition System}\hspace{7.5cm}2010/04 - 2010/05
    \begin{innerlist}
            \item Built a framework including face detection, alignment and
                recognition, adopting Haar Classifier and Subspace LDA, for the purpose of confirming employees' identities;
            \item Led a team of three and was responsible for algorithm design and major part of coding;
            \item This project was supported by Huawei and written in C++ with OpenCV and QT.
    \end{innerlist}


\section{Skills}
Documentation: \LaTeX, HTML, MS Office\\
Programming: Matlab, C++, C, Java, Python, SQL, ASP, PHP, OpenCV, OpenGL\\
Language: Fluent English, Native Mandarin, Japanese (beginner)

\section{Awards}
Scholarship for Outstanding Merits, Zhejiang University, 2008 to 2010\\
Scholarship for Outstanding Students, Zhejiang University, 2008 to 2010\\
Award for Research and Innovation, \engordnumber{1} class, Zhejiang University, 2010\\
Award for Zhejiang University ACM/ICPC Programming Competition, \engordnumber{2} class, 2010\\
Award for Zhejiang University C Programming Language Contest, \engordnumber{1} class, 2009\\
Award for Excellent Student, Zhejiang University, 2008 and 2009\\
\end{document}

%%%%%%%%%%%%%%%%%%%%%%%%%% End CV Document %%%%%%%%%%%%%%%%%%%%%%%%%%%%%
